
\documentclass[a4paper]{article}
\usepackage{mathtools}
\usepackage[utf8x]{inputenc}
\usepackage[pdftex]{graphicx}
\usepackage[usenames,dvipsnames,svgnames,table]{xcolor}
\usepackage{framed}
\usepackage[most]{tcolorbox}
\usepackage[top=4cm, bottom=4cm, left=2cm, right=2cm]{geometry}

\begin{document}
\title{Laboratorio 1}
\author{
        Tabacoff Mila Romana Cécile \\
        Magliona Marco \\
        Lecce Michela \\
        Della Monica Andrea}

\date{\today}
\maketitle

\newpage

\begin{tcolorbox}[breakable,colback=cyan,colframe=cyan]
\section*{Scopo dell'esercitazione}
\end{tcolorbox}

\begin{itemize}
\item Realizzare un generatore con porte logiche
\item Verificare il comportamento di un contatore asincrono
\item Misurare i ritardi del contatore
\item Decodificare uno stato di uscita
\item Verificare la presenza di rimbalzi sui contatti meccanici
\end{itemize}

\begin{tcolorbox}[breakable,colback=cyan,colframe=cyan]
\section*{Progetto->TODO}
\end{tcolorbox}


\begin{tcolorbox}[breakable,colback=cyan,colframe=cyan]
\section*{Montaggio del genatore di onda quadra}
\end{tcolorbox}

Si collega il circuito integrato alla basetta e con due jumper si collega all'alimentazione:
\begin{itemize}
\item piedino 14 a Vcc
\item piedino 7 a GND
\item piedino 1 a ingresso
\item piedino 2 a uscita
\item #JOBFORMARCO #FALLO #ricordati
\end{itemize}
Gli altri piedini sono stati collegati a massa come consigliato, in modo da evitare interferenze esterne.
Si collegano inoltre resistenza e condensatore come da circuito (fig 1); successivamente Va e Vb vengono collegate rispettivamente al + e al - dell'alimentazione.
Si utilizza un cavo sonda-BNC per collegare il CH1 dell'oscilloscopio con GND e resistenza del circuito.

\begin{tcolorbox}[breakable,colback=cyan,colframe=cyan]
\section*{Calcolo capacità del condensatore}
\end{tcolorbox}

\[T'=\tfrac{1}{2f}=\tfrac{1}{20kHz}=R\cdotC\cdot\ln{\tfrac{V_{uL}-V_{s2}}{V_{uL}-V_{s1}}\]
\[C=\tfrac{T'}{R}\cdot\tfrac{1}{\ln{\tfrac{V_{uL}-V_{s2}}{V_{uL}-V_{s1}}}=\tfrac{1}{20kHz\cdot100k\Omega}\cdot\tfrac{1}{\ln{\tfrac{0,001-2,5}{0,0001-1,6}}}= 1,12 * 10^{-9} F = 1,12 nF\]


\begin{tcolorbox}[breakable,colback=cyan,colframe=cyan]
\section*{Misure sul generatore di onda quadra}
\end{tcolorbox}

Visualizzazione:
\begin{itemize}
\item CH1 segnale triangolare dal generatore di segnali a 10kHz
\item CH2 uscita del comparatore
\end{itemize}


Impostazioni oscilloscopio :
\begin{itemize}
\item Sensibilità verticale \(2 V/div\)
\item Sensibilità orizzontale \(25 \mu s/div \) 
\end{itemize}


Si osserva che cambiando l'offset e ponendolo \(<0,9\) (soglia datasheet) il comparatore non commuta: non si osserva l'onda quadra TODO foto 13,14

Superando \(V_{t+ max}=3,15V\) l'onda non è più stabile 

Per completare il circuito si inserisce nuovamente la resistenza da \(100k\Omega\) (come circuito iniziale) e dopo aver scollegato il generatore si osserva: 

Frequenza : \(13,09 kHz\)
  
\begin{center}
  \begin{tabular}{|r|l|l|l|l|}
     \hline
     \multicolumn{3}{|c|}{\(Ingresso (mV)\)} \\
     \hline
     \(960\) & \(800\) & \(1200\) \\
     \hline
   \end{tabular} \\ \\
\[V_I=986,67 \pm 116,24 mV\]


 \begin{tabular}{|r|l|l|l|l|}
     \hline
     \multicolumn{5}{|c|}{\(Uscita (V)\)} \\
     \hline
     \(5,44\) & \(8,32\) & \(6,32\) & \(5,60\) & \(7,60\) \\
     \hline
   \end{tabular} \\ \\

\[V_U=6,65 \pm 0,56 V\]
\end{center}
 L'onda quadra si autogenera anche senza generatore

Si verifica che il trigger è compreso effettivamente tra 1,55 V e 3,15 V, infatti vale 2,4 V  TODO foto 16 

\begin{tcolorbox}[breakable,colback=cyan,colframe=cyan]
\section*{Montaggio del contatore}
\end{tcolorbox}


\begin{tcolorbox}[breakable,colback=cyan,colframe=cyan]
\section*{Misure sul contatore asincrono}
\end{tcolorbox}

\begin{itemize}

\item
Si collega all’ingresso il segnale a onda quadra generato con il circuito precedentemente montato.Si verifica con l’oscilloscopio il corretto funzionamento del divisore (sulle varie uscite sono presenti onde quadre con frequenza via via dimezzata)

\begin{center}
 \begin{tabular}{|r|l|l|l|l|}
     \hline
     \multicolumn{5}{|c|}{Frequenza di ingresso  \((kHz)\)} \\
     \hline
     \(15,43\) & \(15,55\) & \(15,58\) & \(15,63\) & \(15,68\) \\
     \hline
   \end{tabular} \\ \\

\[F_I=15,57 \pm 0,04 kHz\]

\begin{tabular}{|r|l|l|l|l|}
     \hline
     \multicolumn{5}{|c|}{Frequenza di uscita 1  \((kHz)\)} \\
     \hline
     \(7,89\) & \(7,76\) & \(7,77\) & \(7,83\) & \(7,77\) \\
     \hline
   \end{tabular} \\ \\

\[F_{U1}= 7,80 \pm 0,02 kHz\]

\begin{tabular}{|r|l|l|l|l|}
     \hline
     \multicolumn{3}{|c|}{Frequenza di uscita 2  \((kHz)\)} \\
     \hline
     \(3,88\) & \(3,87\) & \(3,89\)  \\
     \hline
   \end{tabular} \\ \\

\[F_{U2}= 3,88 \pm 0,01 kHz\]
\end{center}


\item Si collega all’ingresso di clock una resistenza di pull-up \(R_{pu}= 10 k\Omega\) e un contatto volante con un filo, in modo da poter applicare manualmente il segnale di clock (facendo contatto a massa). Far avanzare a mano il contatore. Si verifica che ad ogni azionamento del contatto il contatore avanza di più passi a causa dei rimbalzi del contatto meccanico (BOUNCE). OK FOTO TANTE

\item Si osserva che il filo collegato all'ingresso del contatore si comporta come un'antenna catturando disturbi dell'ambiente esterno FOTO
\end{itemize}

\LaTeX
\end{document}