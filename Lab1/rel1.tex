
\documentclass{report}
\usepackage{mathtools}
\usepackage[utf8x]{inputenc}

\begin{document}

\chapter{Misurazione di valore efficace e frequenza}

\section{Misurazione del valore efficace}

\begin{itemize}
\item Lettura dell’ampiezza picco-picco \(Vpp= V\)
\item Formula impiegata per il calcolo dell’incertezza: Modello probabilistico \[\bar{n} = \tfrac{1}{m}\sum_{k=1}^m n_k \]
\[s^2 (n_k)= \tfrac{1}{m-1}\sum_{k=1}^m (n_k - \bar{n})^2 \] 
\[s^2 (\bar{n}) = \tfrac{s^2 (n_k)}{m}\]
\item Incertezza: \(\delta{}  Vpp= \)
\item Valore efficace e incertezza \(Veff \)
\end{itemize}

\section{Misurazione di frequenza}
\begin{itemize}
\item Formula incertezza:
\item \(T= s\pm s \)
\item \(f= Hz \pm Hz\)
\end{itemize}

\section{Verifica con multimetro}
\begin{itemize}
\item \(Veff= \pm V \)
\item \(f= Hz \pm Hz \)
\end{itemize}

\chapter{Misurazione del tempo di salita}

\section{ Misurazione 1}
\begin{itemize}
\item Il sistema in misura presenta un disadattamento d’impedenza il cui effetto `e quello di distorcere il fronte di salita del segnale. In queste condizioni il tempo di salita è \(ts1 = ns\)
\item Misurazione in condizioni di adattamento \(ts_2 = ns\) 
\item Tempo di salita introdotto dall’oscilloscopio a causa della sua banda passante \(tso=\tfrac{0.35}{B} = ns\)
\item \(ts= \sqrt{ts_2^2 -tso^2} = ns\)
\end{itemize}

\section{Misurazione 2}
??? la faremo... forse

\begin{enumerate}
\item Frequenza del polo ed effetto sulla misura del tempo di salita
  \begin{itemize}
    \item Capacità totale \(Ctot = pF\)
    \item Resistenza del generatore ”modificato” \(Rg= \)
    \item Frequenza polo \(fp = \tfrac{1}{2\pi \cdot Rg \cdot Ctot} = kHz \)
    \item Tempo di salita dovuto al polo \(tsp= 0.35/fp = ns\)
    \item Verifica sperimentale \(tsp_m = ns\)
    \end{itemize}

\item Per ridurre questo effetto sistematico utilizziamo la sonda compensata al posto del cavo coassiale
 \begin{itemize}
 \item  \(Cs= pf\)
 \end{itemize}
\end{enumerate}

\end{document}






